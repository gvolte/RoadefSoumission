% Exemple d'utilisation de la classe roadef pour le congrès ROADEF 2019 (http://roadef2019.univ-lehavre.fr)

\documentclass{roadef}

\begin{document}


% Le titre du papier
\title{ROADEF 2019, modèle de document \LaTeX{} pour un résumé de \textbf{2} pages maximum}

% Le titre court
\def\shorttitle{Titre court}

% Les auteurs et leur numéro d'affiliation
\author{Marc Sevaux\inst{1}, Maxime Chassaing\inst{1}, André Rossi\inst{2} }


% Les affiliations (par ordre croissant des numéros d'affiliation) séparées par \and
\institute{
Universit\'{e} de Bretagne Sud, Lab-STICC, F-56321 Lorient, France \\
\email{\{marc.sevaux,maxime.chassaing\}@univ-ubs.fr}
\and
Universit\'{e} d'Angers, LERIA, F-49045 Angers, France \\

\email{andre.rossi@univ-angers.fr}
}


% Création de la page de titre
\maketitle
\thispagestyle{empty}

% Les mots-clés
\keywords{recherche opérationnelle, optimisation.}


\section{Introduction}

Ceci est un modèle de document  \LaTeX{} pour un résumé de deux pages maximum dans le cadre du vingtième congrès de la ROADEF qui se déroulera au \textit{Havre les 19, 20 et 21 février 2019}. La limite de deux pages (bibliographie incluse) ne s'applique pas aux soumissions déposées dans la cadre du Prix Jeune Chercheur, pour lesquelles la taille est fixée à quatre pages. Les paragraphes de texte courant seront de style RoadefTexte. Les marges sont de 2,5 cm partout, avec une reliure à gauche de 0,5 cm.


\section{Système de référence}

\subsection{Renvoi à une illustration, tableau ou formule}

Un renvoi à une illustration (figure, graphique...), à un tableau ou à une formule pourra se faire de deux façons différentes : i) la Figure (\ref{logoRoadef}) représente le logo de la ROADEF ou ii) le logo de la ROADEF (voir Figure~\ref{logoRoadef}) est très simple.\footnote{Exemple de pied de page}

\subsection{Renvoi bibliographique}

Les renvois bibliographiques devront être mis entre crochets. Pour des renvois bibliographiques multiples, citer les articles dans l'ordre dans lequel ils apparaissent dans la liste de références (par exemple \cite{toth02,kirkpatrick83}). La liste de références devra être triée dans l'ordre alphabétique du nom de famille du premier auteur. Voici quelques exemples de références : un livre  \cite{toth02}, un article  \cite{kirkpatrick83}.

\subsection{Illustration, formule et légende}

La légende des illustrations devra être positionnée en dessous de l'illustration, comme dans la Figure (\ref{logoRoadef}). Les équations devront être centrées et numérotées avec des chiffres arabes (par exemple Equation \ref{emc}).

\begin{figure}[!ht]
    \begin{center}
        \includegraphics[height=2cm,clip=true]{roadef_logo.eps}
        \caption[Fig]{Logo de la ROADEF}
        \label{logoRoadef}
    \end{center}
\end{figure}

\begin{equation}
E=MC^2\label{emc}
\end{equation}

\begin{theoreme}
  Un exemple de théorème. Les environnements suivants sont également disponibles : remarque, propriété, corollaire, définition, notation, proposition, exemple, preuve.
\end{theoreme}

\subsection{Tableau}

Le titre du tableau devra être positionné sous le tableau (par exemple Tableau \ref{tableau}).

\begin{table}[!ht]
    \begin{center}
        \begin{tabular}{lrr}
            \hline
            & \multicolumn{1}{c}{Colonne 1} & \multicolumn{1}{c}{Colonne 2}\\
            \hline
            Ligne 1 & L1C1 & L1C2\\
            Ligne 2 & L2C1 & L2C2\\
            \hline
        \end{tabular}
        \caption{Exemple de tableau}
    \label{tableau}
    \end{center}
\end{table}

\subsection{Liste}

Voici une liste :

\begin{itemize}
\item remarque
\item propriété
\end{itemize}

\section{Conclusions et perspectives}

Bon courage pour la rédaction !


% La bibliographie

\bibliographystyle{plain}

% Version "on-line" de la bibliographie, mais il est
% également possible d'utiliser un fichier ".bib" et d'utiliser BibTeX


\begin{thebibliography}{2}
\bibitem{toth02}
Paolo Toth and Daniele Vigo.
\newblock \emph{The Vehicle Routing Problem}.
\newblock Monographs on Discrete Mathematics and Applications. Society for Industrial and Applied Mathematics, 2002.
\bibitem{kirkpatrick83}
Scott Kirkpatrick, C~Daniel Gelatt, and Mario~P Vecchi.
\newblock Optimization by simmulated annealing.
\newblock \emph{science}, 220\penalty0 (4598):\penalty0 671--680, 1983.


\end{thebibliography}


\end{document}